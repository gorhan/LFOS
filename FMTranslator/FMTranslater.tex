
\documentclass[]{scrartcl}

\usepackage{listings}
\usepackage{color}

% Title Page
\title{Language Framework for Optimal Schedulers (LFOS)}
\subtitle{Guideline for Users}
\author{G\"{u}ner Orhan}

\definecolor{mygreen}{rgb}{0,0.6,0}
\definecolor{backgroundcolor}{rgb}{1,1,0.95}
\definecolor{mygray}{rgb}{0.5,0.5,0.5}
\definecolor{mymauve}{rgb}{0.58,0,0.82}

\lstset{ %
	backgroundcolor=\color{backgroundcolor},	% choose the background color
	basicstyle=\footnotesize,        			% size of fonts used for the code
	breaklines=true,                 			% automatic line breaking only at whitespace
	captionpos=b,                    			% sets the caption-position to bottom
	commentstyle=\color{mygreen},    			% comment style
	escapeinside={\%*}{*)},          			% if you want to add LaTeX within your code
	keywordstyle=\color{blue},       			% keyword style
	stringstyle=\color{mymauve},    			% string literal style
	showspaces=false,                			% show spaces everywhere adding particular underscores; it overrides 'showstringspaces'
	showstringspaces=false,          			% underline spaces within strings only
	showtabs=false,                  			% show tabs within strings adding particular underscores
	numbers=left,
}

\begin{document}
\maketitle
    
\section{Required Modules}
In order to use LFOS frameowrk API, a programmer should import the required module:

\begin{lstlisting}[language=Python, frame=single, label={lst:moduleImport}, caption={Importing required modules}]
from LFOS.Scheduler.Scheduler import Scheduler
from LFOS.Resource.Resource import *
from LFOS.Task.Task import *
from LFOS.Scheduling.Characteristic.Time import Time
from LFOS.macros import *
\end{lstlisting}
        

\section{Scheduler}
Based on the selected instance, the scheduler instance, namely ``\textsf{sched}'', can be generated by following the following procedures one-by-one:
        
\subsection{Resource Initialization for ``\textsf{cpu1}''}
        
Since some of the task specifications are based on the resources, in the framework, a programmer is expected to define the resources, initially.
As explained in the article, some of the specifications are inevitable for a resource. Therefore, it should be defined for each resource.
Based on the feature model, the following attributes are inevitable for a resource:
\begin{itemize}
    \item \textsc{Capacity} ($\mathcal{C}$): The capacity of the resource are required to determine the maximum amount of capacity which can be utilized
    per time unit.
    \item \textsc{Type} ($\Re$): This attribute categorize the resources based on \emph{Abstractions} and \emph{Identifier}.
    \item \textsc{Mode} ($\mathcal{X}$): The mode of a resource may be either \emph{Shared} or \emph{Exclusive}.
    \item \textsc{Power Consumption} ($\mathcal{V}$): A resource consumes power based on this attribute. The resource is either \emph{Scalable} or not.
    \item \textsc{Objective} ($\mathcal{O}_\alpha$): This attribute is related with the resource-related objectives.
\end{itemize}

The only required specification for the instantiation is \emph{Type} of the resource. The default values for all specification belonging to a resource are
shown in Table \ref{tab:resource_vars_default}.

\begin{table}[!htb]
	\centering
	\begin{tabular}{r || c | c |}
		\hline
		\textbf{Feature Name} & \textbf{Variable Type} & \textbf{Initial Value} \\ \hline \hline
		Capacity ($\mathcal{C})$ & float & $0.0$ \\ \hline
		Type ($\Re$) & LFOS.Resource.Type.ResourceTypeList::Enum & \textsf{proc\_t} \\ \hline
		Mode ($\mathcal{X}$) & LFOS.Resource.Mode.ModeTypeList::Enum & \textsc{cb\_exclusive} \\ \hline
		Power Consumption ($\mathcal{V}$) & LFOS.Resource.Power & \textsf{None} \\ \hline
		Objective ($\mathcal{O}_{\alpha}$) & LFOS.Objective & \textsf{None} \\ \hline
	\end{tabular}
	\caption{Default instance variables of the \emph{Resource} module and their default values.}
	\label{tab:resource_vars_default}
\end{table}

            
For \emph{abstraction}, \textsc{active} is selected. Therefore, you can create the type object using the following code segment:

\begin{lstlisting}[language=Python, frame=single, label={lst:activeType}, caption={Active resource type object instantiation}]
cpu1_t = Type(ResourceTypeList.ACTIVE, 'cpu1_t')
\end{lstlisting}
        
        
According to the specification, a programmer can create the resource giving type object and a name of the resource as arguments to the class method of the
ResourceFactory class shown in Listing \ref{lst:cpu1activeInstantiation}. For active resources, an object belonging to \textsf{TerminalResource} class is instantiated
using factory method pattern to handle the optional feature under \emph{Abstraction} sub-feature.


\begin{lstlisting}[language=Python, frame=single, label={lst:cpu1activeInstantiation}, caption={Active resource instantiation using ResourceFactory class}]
cpu1 = ResourceFactory.create_instance(cpu1_t, 'cpu1')
\end{lstlisting}
        

\subsection{Resource Initialization for ``\textsf{cpu2}''}
        
For \emph{abstraction}, \textsc{active} is selected. Therefore, you can create the type object using the following code segment:

\begin{lstlisting}[language=Python, frame=single, label={lst:activeType}, caption={Active resource type object instantiation}]
cpu2_t = Type(ResourceTypeList.ACTIVE, 'cpu2_t')
\end{lstlisting}
        
        
According to the specification, a programmer can create the resource giving type object and a name of the resource as arguments to the class method of the
ResourceFactory class shown in Listing \ref{lst:cpu2activeInstantiation}. For active resources, an object belonging to \textsf{TerminalResource} class is instantiated
using factory method pattern to handle the optional feature under \emph{Abstraction} sub-feature.


\begin{lstlisting}[language=Python, frame=single, label={lst:cpu2activeInstantiation}, caption={Active resource instantiation using ResourceFactory class}]
cpu2 = ResourceFactory.create_instance(cpu2_t, 'cpu2')
\end{lstlisting}
        

\subsection{Resource Initialization for ``\textsf{cpu1}''}
        
For \emph{abstraction}, \textsc{active} is selected. Therefore, you can create the type object using the following code segment:

\begin{lstlisting}[language=Python, frame=single, label={lst:activeType}, caption={Active resource type object instantiation}]
cpu1_t = Type(ResourceTypeList.ACTIVE, 'cpu1_t')
\end{lstlisting}
        
        
According to the specification, a programmer can create the resource giving type object and a name of the resource as arguments to the class method of the
ResourceFactory class shown in Listing \ref{lst:cpu1activeInstantiation}. For active resources, an object belonging to \textsf{TerminalResource} class is instantiated
using factory method pattern to handle the optional feature under \emph{Abstraction} sub-feature.


\begin{lstlisting}[language=Python, frame=single, label={lst:cpu1activeInstantiation}, caption={Active resource instantiation using ResourceFactory class}]
cpu1 = ResourceFactory.create_instance(cpu1_t, 'cpu1')
\end{lstlisting}
        

\subsection{Resource Initialization for ``\textsf{cpu2}''}
        
For \emph{abstraction}, \textsc{active} is selected. Therefore, you can create the type object using the following code segment:

\begin{lstlisting}[language=Python, frame=single, label={lst:activeType}, caption={Active resource type object instantiation}]
cpu2_t = Type(ResourceTypeList.ACTIVE, 'cpu2_t')
\end{lstlisting}
        
        
According to the specification, a programmer can create the resource giving type object and a name of the resource as arguments to the class method of the
ResourceFactory class shown in Listing \ref{lst:cpu2activeInstantiation}. For active resources, an object belonging to \textsf{TerminalResource} class is instantiated
using factory method pattern to handle the optional feature under \emph{Abstraction} sub-feature.


\begin{lstlisting}[language=Python, frame=single, label={lst:cpu2activeInstantiation}, caption={Active resource instantiation using ResourceFactory class}]
cpu2 = ResourceFactory.create_instance(cpu2_t, 'cpu2')
\end{lstlisting}
        

\subsection{Resource Initialization for ``\textsf{cpu1}''}
        
For \emph{abstraction}, \textsc{active} is selected. Therefore, you can create the type object using the following code segment:

\begin{lstlisting}[language=Python, frame=single, label={lst:activeType}, caption={Active resource type object instantiation}]
cpu1_t = Type(ResourceTypeList.ACTIVE, 'cpu1_t')
\end{lstlisting}
        
        
According to the specification, a programmer can create the resource giving type object and a name of the resource as arguments to the class method of the
ResourceFactory class shown in Listing \ref{lst:cpu1activeInstantiation}. For active resources, an object belonging to \textsf{TerminalResource} class is instantiated
using factory method pattern to handle the optional feature under \emph{Abstraction} sub-feature.


\begin{lstlisting}[language=Python, frame=single, label={lst:cpu1activeInstantiation}, caption={Active resource instantiation using ResourceFactory class}]
cpu1 = ResourceFactory.create_instance(cpu1_t, 'cpu1')
\end{lstlisting}
        

\subsection{Resource Initialization for ``\textsf{cpu2}''}
        
For \emph{abstraction}, \textsc{active} is selected. Therefore, you can create the type object using the following code segment:

\begin{lstlisting}[language=Python, frame=single, label={lst:activeType}, caption={Active resource type object instantiation}]
cpu2_t = Type(ResourceTypeList.ACTIVE, 'cpu2_t')
\end{lstlisting}
        
        
According to the specification, a programmer can create the resource giving type object and a name of the resource as arguments to the class method of the
ResourceFactory class shown in Listing \ref{lst:cpu2activeInstantiation}. For active resources, an object belonging to \textsf{TerminalResource} class is instantiated
using factory method pattern to handle the optional feature under \emph{Abstraction} sub-feature.


\begin{lstlisting}[language=Python, frame=single, label={lst:cpu2activeInstantiation}, caption={Active resource instantiation using ResourceFactory class}]
cpu2 = ResourceFactory.create_instance(cpu2_t, 'cpu2')
\end{lstlisting}
        

\end{document}
